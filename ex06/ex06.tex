\documentclass{article}
\usepackage{amsmath}
\usepackage{amssymb}
\usepackage[svgnames]{xcolor}
\usepackage{graphicx}
\usepackage{enumitem}
\usepackage{multicol}
\usepackage{graphicx}
\usepackage{color}
\usepackage{mathtools}

\title{ML II:  Exercise 2} % Title

\author{Tobias Graf, Philipp Rentzsch, Lina Gundelwein} % Author name

\date{\today}

\begin{document}
	\maketitle
	
	\section{Conditional Independence}
	
	\noindent P(A) = 1st coin flip shows head\\
	P(B) = 2nd coin flip shows head\\
	P(C) = coin is biased (has two heads)
	
	\begin{table}[htb]
		\caption{C=0}
		\centering
		\begin{tabular}{ | c | c | c| c | c|}
			\hline
			A & B & $P(A|C)$ & $P(B|C)$ & $P(A,B|C)$ \\ \hline
			0 & 1 & $\frac{1}{2}$ & $\frac{1}{2}$ & $\frac{1}{2}\cdot\frac{1}{2}$ = $\frac{1}{4}$ \\ \hline 
			0 & 0 & $\frac{1}{2}$ & $\frac{1}{2}$ & $\frac{1}{2}\cdot \frac{1}{2}$ = $\frac{1}{4}$ \\ \hline
			1 & 1 & $\frac{1}{2}$ & $\frac{1}{2}$ & $\frac{1}{2}\cdot\frac{1}{2}$ = $\frac{1}{4}$ \\ \hline
			1 & 0 & $\frac{1}{2}$ & $\frac{1}{2}$ & $\frac{1}{2}\cdot\frac{1}{2}$ = $\frac{1}{4}$ \\  \hline
		\end{tabular}
	\end{table}
	
	\begin{table}[htb]
		\caption{C=1}
		\centering
		\begin{tabular}{ | c | c | c| c | c|}
			\hline
			A & B & $P(A|C)$ & $P(B|C)$ & $P(A,B|C)$ \\ \hline
			0 & 1 &0 &1 & 0 \\ \hline 
			0 & 0 &0 & 0 & 0 \\ \hline
			1 & 1 & 1 & 1 & 1 \\ \hline
			1 & 0 & 1 & 0 & 0 \\  \hline
		\end{tabular}
	\end{table}
	
	\begin{align*}
	P(A) &= P(A|C)P(C)+P(A|\overline{C})P(\overline{C}) \\&= \frac{1}{2}\cdot\frac{1}{2} + 1\cdot\frac{1}{2}=\frac{3}{4}\\
	P(B) &= P(B|C)P(C)+P(B|\overline{C})P(\overline{C}) \\&= \frac{1}{2}\cdot\frac{1}{2} + 1\cdot\frac{1}{2}=\frac{3}{4}\\
	P(A,B) &= P(A, B|C)P(C)+P(A, B|\overline{C})P(\overline{C}) \\&= P(A|C)P(B|C)P(C)+P(A|\overline{C})P(B|\overline{C})P(\overline{C})\\& = \frac{1}{2}\cdot \frac{1}{2}\cdot \frac{1}{2}  + 1\cdot 1\cdot \frac{1}{2}= \frac{5}{8}
	\\
	P(A)P(B) &= \frac{9}{16} \neq P(A,B) = \frac{5}{8}
	\end{align*}
	
	\section{Boy Problem}
	\subsection{Information Theory}
	
	$d$ is the positive Event, either Sunday or Today.
	
	\begin{align*}
	&\frac{p\left(A=b  \wedge B=b  \wedge \left(\left(A=b \wedge A_d = d\right)\vee \left(B=b \wedge B_d = d\right)\right)\right)}{p\left(\left(A=b \wedge A_d = d\right)\vee \left(B=b \wedge B_d = d\right)\right)}\\
	=&\frac{p\left(A=b  \wedge B=b  \wedge \neg\left(A_d =\neg d \wedge B_d = \neg d\right)\right)}{p\left(\neg \left(\neg\left(A=b \wedge A_d = d\right)\wedge \neg\left(B=b \wedge B_d = d\right)\right)\right)}\\
	=&\frac{\frac{1}{2}\frac{1}{2}\left(1-\frac{C-1}{C}\frac{C-1}{C}\right)}{1-\left(1-\frac{1}{2C}\right)^2}\\
	=&\frac{\frac{1}{4}\left(1-\frac{(C-1)^2}{C^2}\right)}{1-\left(1-\frac{1}{2C}\right)^2}\\
	=&\frac{\frac{1}{4}\left(1-\frac{(C^2-2C+1)}{C^2}\right)}{\frac{1}{C}-\frac{1}{4C^2}} \qquad \leftarrow\text{ expand by } \frac{C^2}{C^2} \\
	=&\frac{\frac{1}{4}(2C-1)}{C-\frac{1}{4}}\\
	=& \frac{2C-1}{4C-1}
	\end{align*}
	
	\subsection{Numerical Evaluation}
	
	see associated Jupyter notebook.
	
	\section{Weather Forecast}
	
	a) The missing probabilities are:
	
	\begin{align*}
		P(t_n = \text{Cloudy} | t_{n-1} = \text{Rain}) = \frac{7}{10} \\
		P(t_n = \text{Sunny} | t_{n-1} = \text{Sunny}) = \frac{3}{10} \\
		P(t_n = \text{Sunny} | t_{n-1} = \text{Cloudy}) = \frac{1}{2} \\
	\end{align*}
	
	b) Transitions as a diagram \\
	
	\begin{figure}[htb!]
		\def\svgwidth{0.7\textwidth}
		\input{weather_forecast.pdf_tex} 
	\end{figure}
	
	c \& d) Rain = R, Sunny = S, Cloudy = C
	
	\begin{align*}
		&\vec{p}(t_1) \\
		& =\begin{pmatrix}
		p(t_n = \text{R} | t_{n-1} = \text{R}) \cdot p(t_0 = \text{R}) + p(t_n = \text{R} | t_{n-1} = \text{C}) \cdot p(t_0 = \text{C}) + p(t_n = \text{R} | t_{n-1} = \text{S}) \cdot p(t_0 = \text{S})\\
		p(t_n = \text{C} | t_{n-1} = \text{R}) \cdot p(t_0 = \text{R}) + p(t_n = \text{C} | t_{n-1} = \text{C}) \cdot p(t_0 = \text{C}) + p(t_n = \text{C} | t_{n-1} = \text{S}) \cdot p(t_0 = \text{S}) \\
		p(t_n = \text{S} | t_{n-1} = \text{R}) \cdot p(t_0 = \text{R}) + p(t_n = \text{S} | t_{n-1} = \text{C}) \cdot p(t_0 = \text{C}) + p(t_n = \text{S} | t_{n-1} = \text{S}) \cdot p(t_0 = \text{S})
		\end{pmatrix} \\
		&= M\times\vec{p}(t_0) = 
		\begin{pmatrix}
		0.2 & 0.2 & 0.4 \\
		0.7 & 0.3 & 0.3 \\
		0.1 & 0.5 & 0.3
		\end{pmatrix}
		\times\vec{p}(t_0) =
		\begin{pmatrix}
		0.25 \\ 0.5 \\ 0.25
		\end{pmatrix}
	\end{align*} 
	
	e) we compute
	\begin{align*}
		&\vec{p}(t_n) = M^n \times \vec{p}(t_0) \\
		&\vec{p}(t_{100}) = M^{100} \times \vec{p}(t_0) =
		\begin{pmatrix}
		0.265625 \\
		0.406250 \\
		0.328125
		\end{pmatrix}
	\end{align*}
	
	f)
	for $\lim n \rightarrow \infty$, we solve the equation: $\vec{p}(t_n) = M \times \vec{p}(t_n)$ in a system of linear equations with $p(t_n = \text{R}) = r, p(t_n = \text{C}) = c, p(t_n = \text{S}) = s$:
	\begin{align*}
    &r = 0.2 \cdot r + 0.2 \cdot c + 0.4 \cdot s \\
    &c = 0.7 \cdot r + 0.3 \cdot c + 0.3 \cdot s \\
    &s = 0.1 \cdot r + 0.5 \cdot c + 0.3 \cdot s \\
    &1 = r + c + s
	\end{align*}
	Solving these equations results the solution equal to our result for $\vec{p}(t_{100})$ (due to rounding in e)
	
	
\end{document}
