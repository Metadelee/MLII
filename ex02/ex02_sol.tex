\documentclass{article}
\usepackage{amsmath}
\usepackage{amssymb}
\usepackage[svgnames]{xcolor}
\usepackage{graphicx}
\usepackage{enumitem}
\usepackage{multicol}

\title{ML II:  Exercise 2} % Title

\author{Tobias Graf, Philipp Rentzsch, Lina Gundelwein} % Author name

\date{\today} % Date for the report

\begin{document}
\maketitle 
\section*{2.1 Simple Networks}
\textbf{1. Logical OR on binary input vector:} The weight from the input bias is 0, for all $x_i$ is $w_i = 0$. The binary step function is used as activation.
\begin{figure}[htb]
\def\svgwidth{\textwidth}
%LaTeX with PSTricks extensions
%%Creator: inkscape 0.91
%%Please note this file requires PSTricks extensions
\psset{xunit=.5pt,yunit=.5pt,runit=.5pt}
\begin{pspicture}(744.09448819,1052.36220472)
{
\newrgbcolor{curcolor}{0 0 0}
\pscustom[linewidth=0.92548257,linecolor=curcolor]
{
\newpath
\moveto(292.11582947,900.00000685)
\curveto(292.11582947,875.9510971)(273.25993033,856.45560377)(250,856.45560377)
\curveto(226.74006967,856.45560377)(207.88417053,875.9510971)(207.88417053,900.00000685)
\curveto(207.88417053,924.0489166)(226.74006967,943.54440992)(250,943.54440992)
\curveto(273.25993033,943.54440992)(292.11582947,924.0489166)(292.11582947,900.00000685)
\closepath
}
}
{
\newrgbcolor{curcolor}{0 0 0}
\pscustom[linewidth=1,linecolor=curcolor]
{
\newpath
\moveto(125.71429,820.00000472)
\lineto(214.28571,880.00000472)
}
}
{
\newrgbcolor{curcolor}{0 0 0}
\pscustom[linestyle=none,fillstyle=solid,fillcolor=curcolor]
{
\newpath
\moveto(206.00652335,874.3915229)
\lineto(204.93824142,868.83645551)
\lineto(214.28571,880.00000472)
\lineto(200.45145596,875.45980483)
\lineto(206.00652335,874.3915229)
\closepath
}
}
{
\newrgbcolor{curcolor}{0 0 0}
\pscustom[linewidth=1,linecolor=curcolor]
{
\newpath
\moveto(206.00652335,874.3915229)
\lineto(204.93824142,868.83645551)
\lineto(214.28571,880.00000472)
\lineto(200.45145596,875.45980483)
\lineto(206.00652335,874.3915229)
\closepath
}
}
{
\newrgbcolor{curcolor}{0 0 0}
\pscustom[linewidth=1,linecolor=curcolor]
{
\newpath
\moveto(120,947.04133472)
\lineto(211.42857,921.32704472)
}
}
{
\newrgbcolor{curcolor}{0 0 0}
\pscustom[linestyle=none,fillstyle=solid,fillcolor=curcolor]
{
\newpath
\moveto(201.80206073,924.03450095)
\lineto(196.86847453,921.26687973)
\lineto(211.42857,921.32704472)
\lineto(199.03443951,928.96808715)
\lineto(201.80206073,924.03450095)
\closepath
}
}
{
\newrgbcolor{curcolor}{0 0 0}
\pscustom[linewidth=1,linecolor=curcolor]
{
\newpath
\moveto(201.80206073,924.03450095)
\lineto(196.86847453,921.26687973)
\lineto(211.42857,921.32704472)
\lineto(199.03443951,928.96808715)
\lineto(201.80206073,924.03450095)
\closepath
}
}
{
\newrgbcolor{curcolor}{0 0 0}
\pscustom[linewidth=1,linecolor=curcolor]
{
\newpath
\moveto(292.94424,898.02561472)
\lineto(390.92904,897.01546472)
}
}
{
\newrgbcolor{curcolor}{0 0 0}
\pscustom[linestyle=none,fillstyle=solid,fillcolor=curcolor]
{
\newpath
\moveto(380.92957136,897.11855177)
\lineto(376.88854909,893.15999913)
\lineto(390.92904,897.01546472)
\lineto(376.97101872,901.15957404)
\lineto(380.92957136,897.11855177)
\closepath
}
}
{
\newrgbcolor{curcolor}{0 0 0}
\pscustom[linewidth=1,linecolor=curcolor]
{
\newpath
\moveto(380.92957136,897.11855177)
\lineto(376.88854909,893.15999913)
\lineto(390.92904,897.01546472)
\lineto(376.97101872,901.15957404)
\lineto(380.92957136,897.11855177)
\closepath
}
}
{
\newrgbcolor{curcolor}{0 0 0}
\pscustom[linestyle=none,fillstyle=solid,fillcolor=curcolor]
{
\newpath
\moveto(96.40255737,966.44183272)
\curveto(96.40255737,967.59824548)(96.16329956,968.50400719)(95.68478394,969.15911787)
\curveto(95.21196493,969.81992516)(94.55970256,970.15032881)(93.72799683,970.15032881)
\curveto(92.8962911,970.15032881)(92.24118042,969.81992516)(91.76266479,969.15911787)
\curveto(91.28984578,968.50400719)(91.05343628,967.59824548)(91.05343628,966.44183272)
\curveto(91.05343628,965.28541995)(91.28984578,964.37680993)(91.76266479,963.71600264)
\curveto(92.24118042,963.06089196)(92.8962911,962.73333662)(93.72799683,962.73333662)
\curveto(94.55970256,962.73333662)(95.21196493,963.06089196)(95.68478394,963.71600264)
\curveto(96.16329956,964.37680993)(96.40255737,965.28541995)(96.40255737,966.44183272)
\closepath
\moveto(91.05343628,969.78289717)
\curveto(91.38383993,970.35255863)(91.79969279,970.77410811)(92.30099487,971.04754561)
\curveto(92.80799357,971.32667972)(93.41183472,971.46624678)(94.11251831,971.46624678)
\curveto(95.27462769,971.46624678)(96.2174174,971.004821)(96.94088745,970.08196943)
\curveto(97.67005412,969.15911787)(98.03463745,967.94573897)(98.03463745,966.44183272)
\curveto(98.03463745,964.93792647)(97.67005412,963.72454756)(96.94088745,962.801696)
\curveto(96.2174174,961.87884443)(95.27462769,961.41741865)(94.11251831,961.41741865)
\curveto(93.41183472,961.41741865)(92.80799357,961.5541374)(92.30099487,961.8275749)
\curveto(91.79969279,962.10670902)(91.38383993,962.5311068)(91.05343628,963.10076826)
\lineto(91.05343628,961.66522139)
\lineto(89.47262573,961.66522139)
\lineto(89.47262573,974.96111982)
\lineto(91.05343628,974.96111982)
\lineto(91.05343628,969.78289717)
\closepath
}
}
{
\newrgbcolor{curcolor}{0 0 0}
\pscustom[linestyle=none,fillstyle=solid,fillcolor=curcolor]
{
\newpath
\moveto(100.64083862,971.23553389)
\lineto(102.21310425,971.23553389)
\lineto(102.21310425,961.66522139)
\lineto(100.64083862,961.66522139)
\lineto(100.64083862,971.23553389)
\closepath
\moveto(100.64083862,974.96111982)
\lineto(102.21310425,974.96111982)
\lineto(102.21310425,972.97015303)
\lineto(100.64083862,972.97015303)
\lineto(100.64083862,974.96111982)
\closepath
}
}
{
\newrgbcolor{curcolor}{0 0 0}
\pscustom[linestyle=none,fillstyle=solid,fillcolor=curcolor]
{
\newpath
\moveto(109.84371948,966.4760124)
\curveto(108.57337443,966.4760124)(107.69324748,966.33074873)(107.20333862,966.04022139)
\curveto(106.71342977,965.74969404)(106.46847534,965.25408857)(106.46847534,964.55340498)
\curveto(106.46847534,963.99513675)(106.65076701,963.55080081)(107.01535034,963.22039717)
\curveto(107.38563029,962.89569014)(107.88693237,962.73333662)(108.51925659,962.73333662)
\curveto(109.39083862,962.73333662)(110.08867391,963.04095381)(110.61276245,963.65618818)
\curveto(111.14254761,964.27711917)(111.40744019,965.10027998)(111.40744019,966.12567061)
\lineto(111.40744019,966.4760124)
\lineto(109.84371948,966.4760124)
\closepath
\moveto(112.97970581,967.12542647)
\lineto(112.97970581,961.66522139)
\lineto(111.40744019,961.66522139)
\lineto(111.40744019,963.11785811)
\curveto(111.04855347,962.53680342)(110.60136922,962.10670902)(110.06588745,961.8275749)
\curveto(109.53040568,961.5541374)(108.875295,961.41741865)(108.10055542,961.41741865)
\curveto(107.12073771,961.41741865)(106.34030151,961.69085615)(105.75924683,962.23773115)
\curveto(105.18388875,962.79030277)(104.89620972,963.52801436)(104.89620972,964.45086592)
\curveto(104.89620972,965.52752607)(105.25509644,966.33929365)(105.97286987,966.88616865)
\curveto(106.69633993,967.43304365)(107.77300008,967.70648115)(109.20285034,967.70648115)
\lineto(111.40744019,967.70648115)
\lineto(111.40744019,967.86028975)
\curveto(111.40744019,968.5837598)(111.16818237,969.14202803)(110.68966675,969.53509443)
\curveto(110.21684774,969.93385745)(109.55034383,970.13323897)(108.69015503,970.13323897)
\curveto(108.14328003,970.13323897)(107.61064657,970.0677279)(107.09225464,969.93670576)
\curveto(106.57386271,969.80568363)(106.07540894,969.60915042)(105.59689331,969.34710615)
\lineto(105.59689331,970.79974287)
\curveto(106.17225138,971.02191084)(106.73051961,971.18711266)(107.271698,971.29534834)
\curveto(107.81287638,971.40928063)(108.33981323,971.46624678)(108.85250854,971.46624678)
\curveto(110.23678589,971.46624678)(111.27072144,971.10736006)(111.95431519,970.38958662)
\curveto(112.63790894,969.67181318)(112.97970581,968.5837598)(112.97970581,967.12542647)
\closepath
}
}
{
\newrgbcolor{curcolor}{0 0 0}
\pscustom[linestyle=none,fillstyle=solid,fillcolor=curcolor]
{
\newpath
\moveto(122.32785034,970.95355147)
\lineto(122.32785034,969.46673506)
\curveto(121.8835144,969.69459964)(121.42208862,969.86549808)(120.943573,969.97943037)
\curveto(120.46505737,970.09336266)(119.9694519,970.15032881)(119.45675659,970.15032881)
\curveto(118.67632039,970.15032881)(118.08956909,970.0306999)(117.69650269,969.79144209)
\curveto(117.30913289,969.55218428)(117.115448,969.19329756)(117.115448,968.71478193)
\curveto(117.115448,968.3501986)(117.25501506,968.06251956)(117.53414917,967.85174482)
\curveto(117.81328328,967.6466667)(118.37439982,967.4501335)(119.21749878,967.26214522)
\lineto(119.75582886,967.14251631)
\curveto(120.87236532,966.9032585)(121.66419474,966.56430993)(122.13131714,966.12567061)
\curveto(122.60413615,965.6927279)(122.84054565,965.08603844)(122.84054565,964.30560225)
\curveto(122.84054565,963.41693037)(122.48735555,962.71339847)(121.78097534,962.19500654)
\curveto(121.08029175,961.67661462)(120.11471558,961.41741865)(118.88424683,961.41741865)
\curveto(118.37155151,961.41741865)(117.83606974,961.46868818)(117.27780151,961.57122725)
\curveto(116.7252299,961.66806969)(116.1413269,961.81618167)(115.52609253,962.01556318)
\lineto(115.52609253,963.63909834)
\curveto(116.10714722,963.33717777)(116.67965698,963.10931318)(117.24362183,962.95550459)
\curveto(117.80758667,962.80739261)(118.3658549,962.73333662)(118.91842651,962.73333662)
\curveto(119.65898641,962.73333662)(120.22864787,962.85866214)(120.62741089,963.10931318)
\curveto(121.02617391,963.36566084)(121.22555542,963.72454756)(121.22555542,964.18597334)
\curveto(121.22555542,964.61321943)(121.08029175,964.94077477)(120.7897644,965.16863936)
\curveto(120.50493368,965.39650394)(119.87545776,965.6158236)(118.90133667,965.82659834)
\lineto(118.35446167,965.95477217)
\curveto(117.38034058,966.15985029)(116.67680868,966.4731641)(116.24386597,966.89471357)
\curveto(115.81092326,967.32195967)(115.5944519,967.90586266)(115.5944519,968.64642256)
\curveto(115.5944519,969.54648766)(115.91346232,970.24147464)(116.55148315,970.7313835)
\curveto(117.18950399,971.22129235)(118.09526571,971.46624678)(119.26876831,971.46624678)
\curveto(119.849823,971.46624678)(120.396698,971.42352217)(120.90939331,971.33807295)
\curveto(121.42208862,971.25262373)(121.89490763,971.1244499)(122.32785034,970.95355147)
\closepath
}
}
{
\newrgbcolor{curcolor}{0 0 0}
\pscustom[linestyle=none,fillstyle=solid,fillcolor=curcolor]
{
\newpath
\moveto(100.38874054,810.32434767)
\lineto(98.43561554,810.32434767)
\lineto(98.41608429,816.20325392)
\curveto(97.04889679,816.22929559)(95.68170929,816.38554559)(94.31452179,816.67200392)
\curveto(92.94733429,816.97148309)(91.57363637,817.41419142)(90.19342804,818.00012892)
\lineto(90.19342804,821.51575392)
\curveto(91.52155304,820.68242059)(92.86269887,820.05091017)(94.21686554,819.62122267)
\curveto(95.58405304,819.204556)(96.99030304,818.98971225)(98.43561554,818.97669142)
\lineto(98.43561554,827.88294142)
\curveto(95.55801137,828.35169142)(93.46165721,829.14596225)(92.14655304,830.26575392)
\curveto(90.84446971,831.38554559)(90.19342804,832.92200392)(90.19342804,834.87512892)
\curveto(90.19342804,836.99752475)(90.90306346,838.67070184)(92.32233429,839.89466017)
\curveto(93.74160512,841.1186185)(95.77936554,841.8217435)(98.43561554,842.00403517)
\lineto(98.43561554,846.59387892)
\lineto(100.38874054,846.59387892)
\lineto(100.38874054,842.06262892)
\curveto(101.59967804,842.01054559)(102.77155304,841.88033725)(103.90436554,841.67200392)
\curveto(105.03717804,841.47669142)(106.14394887,841.20325392)(107.22467804,840.85169142)
\lineto(107.22467804,837.43372267)
\curveto(106.14394887,837.98059767)(105.03066762,838.40377475)(103.88483429,838.70325392)
\curveto(102.75202179,839.00273309)(101.58665721,839.17851434)(100.38874054,839.23059767)
\lineto(100.38874054,830.89075392)
\curveto(103.34446971,830.43502475)(105.51894887,829.62122267)(106.91217804,828.44934767)
\curveto(108.30540721,827.27747267)(109.00202179,825.67591017)(109.00202179,823.64466017)
\curveto(109.00202179,821.44413934)(108.25983429,819.70585809)(106.77545929,818.42981642)
\curveto(105.30410512,817.16679559)(103.17519887,816.43762892)(100.38874054,816.24231642)
\lineto(100.38874054,810.32434767)
\closepath
\moveto(98.43561554,831.24231642)
\lineto(98.43561554,839.25012892)
\curveto(96.92519887,839.08085809)(95.77285512,838.65117059)(94.97858429,837.96106642)
\curveto(94.18431346,837.27096225)(93.78717804,836.3529935)(93.78717804,835.20716017)
\curveto(93.78717804,834.0873685)(94.15176137,833.21497267)(94.88092804,832.58997267)
\curveto(95.62311554,831.96497267)(96.80801137,831.51575392)(98.43561554,831.24231642)
\closepath
\moveto(100.38874054,827.49231642)
\lineto(100.38874054,819.03528517)
\curveto(102.04238637,819.25663934)(103.28587596,819.72538934)(104.11920929,820.44153517)
\curveto(104.96556346,821.157681)(105.38874054,822.10169142)(105.38874054,823.27356642)
\curveto(105.38874054,824.41939975)(104.98509471,825.33085809)(104.17780304,826.00794142)
\curveto(103.38353221,826.68502475)(102.12051137,827.17981642)(100.38874054,827.49231642)
\closepath
}
}
{
\newrgbcolor{curcolor}{0 0 0}
\pscustom[linestyle=none,fillstyle=solid,fillcolor=curcolor]
{
\newpath
\moveto(134.29499054,838.07825392)
\lineto(126.38483429,827.43372267)
\lineto(134.70514679,816.20325392)
\lineto(130.46686554,816.20325392)
\lineto(124.09967804,824.79700392)
\lineto(117.73249054,816.20325392)
\lineto(113.49420929,816.20325392)
\lineto(121.99030304,827.64856642)
\lineto(114.21686554,838.07825392)
\lineto(118.45514679,838.07825392)
\lineto(124.25592804,830.28528517)
\lineto(130.05670929,838.07825392)
\lineto(134.29499054,838.07825392)
\closepath
}
}
{
\newrgbcolor{curcolor}{0 0 0}
\pscustom[linestyle=none,fillstyle=solid,fillcolor=curcolor]
{
\newpath
\moveto(156.40436554,809.56262892)
\lineto(156.40436554,806.76966017)
\lineto(135.62311554,806.76966017)
\lineto(135.62311554,809.56262892)
\lineto(156.40436554,809.56262892)
\closepath
}
}
{
\newrgbcolor{curcolor}{0 0 0}
\pscustom[linestyle=none,fillstyle=solid,fillcolor=curcolor]
{
\newpath
\moveto(159.78327179,838.07825392)
\lineto(163.37702179,838.07825392)
\lineto(163.37702179,816.20325392)
\lineto(159.78327179,816.20325392)
\lineto(159.78327179,838.07825392)
\closepath
\moveto(159.78327179,846.59387892)
\lineto(163.37702179,846.59387892)
\lineto(163.37702179,842.04309767)
\lineto(159.78327179,842.04309767)
\lineto(159.78327179,846.59387892)
\closepath
}
}
{
\newrgbcolor{curcolor}{0 0 0}
\pscustom[linestyle=none,fillstyle=solid,fillcolor=curcolor]
{
\newpath
\moveto(180.62311554,810.32434767)
\lineto(178.66999054,810.32434767)
\lineto(178.65045929,816.20325392)
\curveto(177.28327179,816.22929559)(175.91608429,816.38554559)(174.54889679,816.67200392)
\curveto(173.18170929,816.97148309)(171.80801137,817.41419142)(170.42780304,818.00012892)
\lineto(170.42780304,821.51575392)
\curveto(171.75592804,820.68242059)(173.09707387,820.05091017)(174.45124054,819.62122267)
\curveto(175.81842804,819.204556)(177.22467804,818.98971225)(178.66999054,818.97669142)
\lineto(178.66999054,827.88294142)
\curveto(175.79238637,828.35169142)(173.69603221,829.14596225)(172.38092804,830.26575392)
\curveto(171.07884471,831.38554559)(170.42780304,832.92200392)(170.42780304,834.87512892)
\curveto(170.42780304,836.99752475)(171.13743846,838.67070184)(172.55670929,839.89466017)
\curveto(173.97598012,841.1186185)(176.01374054,841.8217435)(178.66999054,842.00403517)
\lineto(178.66999054,846.59387892)
\lineto(180.62311554,846.59387892)
\lineto(180.62311554,842.06262892)
\curveto(181.83405304,842.01054559)(183.00592804,841.88033725)(184.13874054,841.67200392)
\curveto(185.27155304,841.47669142)(186.37832387,841.20325392)(187.45905304,840.85169142)
\lineto(187.45905304,837.43372267)
\curveto(186.37832387,837.98059767)(185.26504262,838.40377475)(184.11920929,838.70325392)
\curveto(182.98639679,839.00273309)(181.82103221,839.17851434)(180.62311554,839.23059767)
\lineto(180.62311554,830.89075392)
\curveto(183.57884471,830.43502475)(185.75332387,829.62122267)(187.14655304,828.44934767)
\curveto(188.53978221,827.27747267)(189.23639679,825.67591017)(189.23639679,823.64466017)
\curveto(189.23639679,821.44413934)(188.49420929,819.70585809)(187.00983429,818.42981642)
\curveto(185.53848012,817.16679559)(183.40957387,816.43762892)(180.62311554,816.24231642)
\lineto(180.62311554,810.32434767)
\closepath
\moveto(178.66999054,831.24231642)
\lineto(178.66999054,839.25012892)
\curveto(177.15957387,839.08085809)(176.00723012,838.65117059)(175.21295929,837.96106642)
\curveto(174.41868846,837.27096225)(174.02155304,836.3529935)(174.02155304,835.20716017)
\curveto(174.02155304,834.0873685)(174.38613637,833.21497267)(175.11530304,832.58997267)
\curveto(175.85749054,831.96497267)(177.04238637,831.51575392)(178.66999054,831.24231642)
\closepath
\moveto(180.62311554,827.49231642)
\lineto(180.62311554,819.03528517)
\curveto(182.27676137,819.25663934)(183.52025096,819.72538934)(184.35358429,820.44153517)
\curveto(185.19993846,821.157681)(185.62311554,822.10169142)(185.62311554,823.27356642)
\curveto(185.62311554,824.41939975)(185.21946971,825.33085809)(184.41217804,826.00794142)
\curveto(183.61790721,826.68502475)(182.35488637,827.17981642)(180.62311554,827.49231642)
\closepath
}
}
{
\newrgbcolor{curcolor}{0 0 0}
\pscustom[linestyle=none,fillstyle=solid,fillcolor=curcolor]
{
\newpath
\moveto(168.06896973,944.67463606)
\lineto(166.11584473,944.67463606)
\lineto(166.09631348,950.55354231)
\curveto(164.72912598,950.57958398)(163.36193848,950.73583398)(161.99475098,951.02229231)
\curveto(160.62756348,951.32177148)(159.25386556,951.76447981)(157.87365723,952.35041731)
\lineto(157.87365723,955.86604231)
\curveto(159.20178223,955.03270898)(160.54292806,954.40119856)(161.89709473,953.97151106)
\curveto(163.26428223,953.55484439)(164.67053223,953.34000064)(166.11584473,953.32697981)
\lineto(166.11584473,962.23322981)
\curveto(163.23824056,962.70197981)(161.14188639,963.49625064)(159.82678223,964.61604231)
\curveto(158.52469889,965.73583398)(157.87365723,967.27229231)(157.87365723,969.22541731)
\curveto(157.87365723,971.34781314)(158.58329264,973.02099023)(160.00256348,974.24494856)
\curveto(161.42183431,975.46890689)(163.45959473,976.17203189)(166.11584473,976.35432356)
\lineto(166.11584473,980.94416731)
\lineto(168.06896973,980.94416731)
\lineto(168.06896973,976.41291731)
\curveto(169.27990723,976.36083398)(170.45178223,976.23062564)(171.58459473,976.02229231)
\curveto(172.71740723,975.82697981)(173.82417806,975.55354231)(174.90490723,975.20197981)
\lineto(174.90490723,971.78401106)
\curveto(173.82417806,972.33088606)(172.71089681,972.75406314)(171.56506348,973.05354231)
\curveto(170.43225098,973.35302148)(169.26688639,973.52880273)(168.06896973,973.58088606)
\lineto(168.06896973,965.24104231)
\curveto(171.02469889,964.78531314)(173.19917806,963.97151106)(174.59240723,962.79963606)
\curveto(175.98563639,961.62776106)(176.68225098,960.02619856)(176.68225098,957.99494856)
\curveto(176.68225098,955.79442773)(175.94006348,954.05614648)(174.45568848,952.78010481)
\curveto(172.98433431,951.51708398)(170.85542806,950.78791731)(168.06896973,950.59260481)
\lineto(168.06896973,944.67463606)
\closepath
\moveto(166.11584473,965.59260481)
\lineto(166.11584473,973.60041731)
\curveto(164.60542806,973.43114648)(163.45308431,973.00145898)(162.65881348,972.31135481)
\curveto(161.86454264,971.62125064)(161.46740723,970.70328189)(161.46740723,969.55744856)
\curveto(161.46740723,968.43765689)(161.83199056,967.56526106)(162.56115723,966.94026106)
\curveto(163.30334473,966.31526106)(164.48824056,965.86604231)(166.11584473,965.59260481)
\closepath
\moveto(168.06896973,961.84260481)
\lineto(168.06896973,953.38557356)
\curveto(169.72261556,953.60692773)(170.96610514,954.07567773)(171.79943848,954.79182356)
\curveto(172.64579264,955.50796939)(173.06896973,956.45197981)(173.06896973,957.62385481)
\curveto(173.06896973,958.76968814)(172.66532389,959.68114648)(171.85803223,960.35822981)
\curveto(171.06376139,961.03531314)(169.80074056,961.53010481)(168.06896973,961.84260481)
\closepath
}
}
{
\newrgbcolor{curcolor}{0 0 0}
\pscustom[linestyle=none,fillstyle=solid,fillcolor=curcolor]
{
\newpath
\moveto(181.70178223,972.42854231)
\lineto(185.29553223,972.42854231)
\lineto(189.78771973,955.35822981)
\lineto(194.26037598,972.42854231)
\lineto(198.49865723,972.42854231)
\lineto(202.99084473,955.35822981)
\lineto(207.46350098,972.42854231)
\lineto(211.05725098,972.42854231)
\lineto(205.33459473,950.55354231)
\lineto(201.09631348,950.55354231)
\lineto(196.38928223,968.48322981)
\lineto(191.66271973,950.55354231)
\lineto(187.42443848,950.55354231)
\lineto(181.70178223,972.42854231)
\closepath
}
}
{
\newrgbcolor{curcolor}{0 0 0}
\pscustom[linestyle=none,fillstyle=solid,fillcolor=curcolor]
{
\newpath
\moveto(233.14709473,943.91291731)
\lineto(233.14709473,941.11994856)
\lineto(212.36584473,941.11994856)
\lineto(212.36584473,943.91291731)
\lineto(233.14709473,943.91291731)
\closepath
}
}
{
\newrgbcolor{curcolor}{0 0 0}
\pscustom[linestyle=none,fillstyle=solid,fillcolor=curcolor]
{
\newpath
\moveto(245.47131348,977.11604231)
\curveto(243.44006348,977.11604231)(241.91011556,976.11343814)(240.88146973,974.10822981)
\curveto(239.86584473,972.11604231)(239.35803223,969.11474023)(239.35803223,965.10432356)
\curveto(239.35803223,961.10692773)(239.86584473,958.10562564)(240.88146973,956.10041731)
\curveto(241.91011556,954.10822981)(243.44006348,953.11213606)(245.47131348,953.11213606)
\curveto(247.51558431,953.11213606)(249.04553223,954.10822981)(250.06115723,956.10041731)
\curveto(251.08980306,958.10562564)(251.60412598,961.10692773)(251.60412598,965.10432356)
\curveto(251.60412598,969.11474023)(251.08980306,972.11604231)(250.06115723,974.10822981)
\curveto(249.04553223,976.11343814)(247.51558431,977.11604231)(245.47131348,977.11604231)
\closepath
\moveto(245.47131348,980.24104231)
\curveto(248.73954264,980.24104231)(251.23303223,978.94546939)(252.95178223,976.35432356)
\curveto(254.68355306,973.77619856)(255.54943848,970.02619856)(255.54943848,965.10432356)
\curveto(255.54943848,960.19546939)(254.68355306,956.44546939)(252.95178223,953.85432356)
\curveto(251.23303223,951.27619856)(248.73954264,949.98713606)(245.47131348,949.98713606)
\curveto(242.20308431,949.98713606)(239.70308431,951.27619856)(237.97131348,953.85432356)
\curveto(236.25256348,956.44546939)(235.39318848,960.19546939)(235.39318848,965.10432356)
\curveto(235.39318848,970.02619856)(236.25256348,973.77619856)(237.97131348,976.35432356)
\curveto(239.70308431,978.94546939)(242.20308431,980.24104231)(245.47131348,980.24104231)
\closepath
}
}
{
\newrgbcolor{curcolor}{0 0 0}
\pscustom[linestyle=none,fillstyle=solid,fillcolor=curcolor]
{
\newpath
\moveto(275.19787598,968.71760481)
\lineto(300.23693848,968.71760481)
\lineto(300.23693848,965.43635481)
\lineto(275.19787598,965.43635481)
\lineto(275.19787598,968.71760481)
\closepath
\moveto(275.19787598,960.74885481)
\lineto(300.23693848,960.74885481)
\lineto(300.23693848,957.42854231)
\lineto(275.19787598,957.42854231)
\lineto(275.19787598,960.74885481)
\closepath
}
}
{
\newrgbcolor{curcolor}{0 0 0}
\pscustom[linestyle=none,fillstyle=solid,fillcolor=curcolor]
{
\newpath
\moveto(329.92443848,977.11604231)
\curveto(327.89318848,977.11604231)(326.36324056,976.11343814)(325.33459473,974.10822981)
\curveto(324.31896973,972.11604231)(323.81115723,969.11474023)(323.81115723,965.10432356)
\curveto(323.81115723,961.10692773)(324.31896973,958.10562564)(325.33459473,956.10041731)
\curveto(326.36324056,954.10822981)(327.89318848,953.11213606)(329.92443848,953.11213606)
\curveto(331.96870931,953.11213606)(333.49865723,954.10822981)(334.51428223,956.10041731)
\curveto(335.54292806,958.10562564)(336.05725098,961.10692773)(336.05725098,965.10432356)
\curveto(336.05725098,969.11474023)(335.54292806,972.11604231)(334.51428223,974.10822981)
\curveto(333.49865723,976.11343814)(331.96870931,977.11604231)(329.92443848,977.11604231)
\closepath
\moveto(329.92443848,980.24104231)
\curveto(333.19266764,980.24104231)(335.68615723,978.94546939)(337.40490723,976.35432356)
\curveto(339.13667806,973.77619856)(340.00256348,970.02619856)(340.00256348,965.10432356)
\curveto(340.00256348,960.19546939)(339.13667806,956.44546939)(337.40490723,953.85432356)
\curveto(335.68615723,951.27619856)(333.19266764,949.98713606)(329.92443848,949.98713606)
\curveto(326.65620931,949.98713606)(324.15620931,951.27619856)(322.42443848,953.85432356)
\curveto(320.70568848,956.44546939)(319.84631348,960.19546939)(319.84631348,965.10432356)
\curveto(319.84631348,970.02619856)(320.70568848,973.77619856)(322.42443848,976.35432356)
\curveto(324.15620931,978.94546939)(326.65620931,980.24104231)(329.92443848,980.24104231)
\closepath
}
}
{
\newrgbcolor{curcolor}{0 0 0}
\pscustom[linestyle=none,fillstyle=solid,fillcolor=curcolor]
{
\newpath
\moveto(356.19396973,944.67463606)
\lineto(354.24084473,944.67463606)
\lineto(354.22131348,950.55354231)
\curveto(352.85412598,950.57958398)(351.48693848,950.73583398)(350.11975098,951.02229231)
\curveto(348.75256348,951.32177148)(347.37886556,951.76447981)(345.99865723,952.35041731)
\lineto(345.99865723,955.86604231)
\curveto(347.32678223,955.03270898)(348.66792806,954.40119856)(350.02209473,953.97151106)
\curveto(351.38928223,953.55484439)(352.79553223,953.34000064)(354.24084473,953.32697981)
\lineto(354.24084473,962.23322981)
\curveto(351.36324056,962.70197981)(349.26688639,963.49625064)(347.95178223,964.61604231)
\curveto(346.64969889,965.73583398)(345.99865723,967.27229231)(345.99865723,969.22541731)
\curveto(345.99865723,971.34781314)(346.70829264,973.02099023)(348.12756348,974.24494856)
\curveto(349.54683431,975.46890689)(351.58459473,976.17203189)(354.24084473,976.35432356)
\lineto(354.24084473,980.94416731)
\lineto(356.19396973,980.94416731)
\lineto(356.19396973,976.41291731)
\curveto(357.40490723,976.36083398)(358.57678223,976.23062564)(359.70959473,976.02229231)
\curveto(360.84240723,975.82697981)(361.94917806,975.55354231)(363.02990723,975.20197981)
\lineto(363.02990723,971.78401106)
\curveto(361.94917806,972.33088606)(360.83589681,972.75406314)(359.69006348,973.05354231)
\curveto(358.55725098,973.35302148)(357.39188639,973.52880273)(356.19396973,973.58088606)
\lineto(356.19396973,965.24104231)
\curveto(359.14969889,964.78531314)(361.32417806,963.97151106)(362.71740723,962.79963606)
\curveto(364.11063639,961.62776106)(364.80725098,960.02619856)(364.80725098,957.99494856)
\curveto(364.80725098,955.79442773)(364.06506348,954.05614648)(362.58068848,952.78010481)
\curveto(361.10933431,951.51708398)(358.98042806,950.78791731)(356.19396973,950.59260481)
\lineto(356.19396973,944.67463606)
\closepath
\moveto(354.24084473,965.59260481)
\lineto(354.24084473,973.60041731)
\curveto(352.73042806,973.43114648)(351.57808431,973.00145898)(350.78381348,972.31135481)
\curveto(349.98954264,971.62125064)(349.59240723,970.70328189)(349.59240723,969.55744856)
\curveto(349.59240723,968.43765689)(349.95699056,967.56526106)(350.68615723,966.94026106)
\curveto(351.42834473,966.31526106)(352.61324056,965.86604231)(354.24084473,965.59260481)
\closepath
\moveto(356.19396973,961.84260481)
\lineto(356.19396973,953.38557356)
\curveto(357.84761556,953.60692773)(359.09110514,954.07567773)(359.92443848,954.79182356)
\curveto(360.77079264,955.50796939)(361.19396973,956.45197981)(361.19396973,957.62385481)
\curveto(361.19396973,958.76968814)(360.79032389,959.68114648)(359.98303223,960.35822981)
\curveto(359.18876139,961.03531314)(357.92574056,961.53010481)(356.19396973,961.84260481)
\closepath
}
}
{
\newrgbcolor{curcolor}{0 0 0}
\pscustom[linestyle=none,fillstyle=solid,fillcolor=curcolor]
{
\newpath
\moveto(205.4446106,841.63907545)
\lineto(203.4914856,841.63907545)
\lineto(203.47195435,847.5179817)
\curveto(202.10476685,847.54402337)(200.73757935,847.70027337)(199.37039185,847.9867317)
\curveto(198.00320435,848.28621087)(196.62950643,848.7289192)(195.2492981,849.3148567)
\lineto(195.2492981,852.8304817)
\curveto(196.5774231,851.99714837)(197.91856893,851.36563795)(199.2727356,850.93595045)
\curveto(200.6399231,850.51928379)(202.0461731,850.30444004)(203.4914856,850.2914192)
\lineto(203.4914856,859.1976692)
\curveto(200.61388143,859.6664192)(198.51752726,860.46069004)(197.2024231,861.5804817)
\curveto(195.90033976,862.70027337)(195.2492981,864.2367317)(195.2492981,866.1898567)
\curveto(195.2492981,868.31225254)(195.95893351,869.98542962)(197.37820435,871.20938795)
\curveto(198.79747518,872.43334629)(200.8352356,873.13647129)(203.4914856,873.31876295)
\lineto(203.4914856,877.9086067)
\lineto(205.4446106,877.9086067)
\lineto(205.4446106,873.3773567)
\curveto(206.6555481,873.32527337)(207.8274231,873.19506504)(208.9602356,872.9867317)
\curveto(210.0930481,872.7914192)(211.19981893,872.5179817)(212.2805481,872.1664192)
\lineto(212.2805481,868.74845045)
\curveto(211.19981893,869.29532545)(210.08653768,869.71850254)(208.94070435,870.0179817)
\curveto(207.80789185,870.31746087)(206.64252726,870.49324212)(205.4446106,870.54532545)
\lineto(205.4446106,862.2054817)
\curveto(208.40033976,861.74975254)(210.57481893,860.93595045)(211.9680481,859.76407545)
\curveto(213.36127726,858.59220045)(214.05789185,856.99063795)(214.05789185,854.95938795)
\curveto(214.05789185,852.75886712)(213.31570435,851.02058587)(211.83132935,849.7445442)
\curveto(210.35997518,848.48152337)(208.23106893,847.7523567)(205.4446106,847.5570442)
\lineto(205.4446106,841.63907545)
\closepath
\moveto(203.4914856,862.5570442)
\lineto(203.4914856,870.5648567)
\curveto(201.98106893,870.39558587)(200.82872518,869.96589837)(200.03445435,869.2757942)
\curveto(199.24018351,868.58569004)(198.8430481,867.66772129)(198.8430481,866.52188795)
\curveto(198.8430481,865.40209629)(199.20763143,864.52970045)(199.9367981,863.90470045)
\curveto(200.6789856,863.27970045)(201.86388143,862.8304817)(203.4914856,862.5570442)
\closepath
\moveto(205.4446106,858.8070442)
\lineto(205.4446106,850.35001295)
\curveto(207.09825643,850.57136712)(208.34174601,851.04011712)(209.17507935,851.75626295)
\curveto(210.02143351,852.47240879)(210.4446106,853.4164192)(210.4446106,854.5882942)
\curveto(210.4446106,855.73412754)(210.04096476,856.64558587)(209.2336731,857.3226692)
\curveto(208.43940226,857.99975254)(207.17638143,858.4945442)(205.4446106,858.8070442)
\closepath
}
}
{
\newrgbcolor{curcolor}{0 0 0}
\pscustom[linestyle=none,fillstyle=solid,fillcolor=curcolor]
{
\newpath
\moveto(219.0774231,869.3929817)
\lineto(222.6711731,869.3929817)
\lineto(227.1633606,852.3226692)
\lineto(231.63601685,869.3929817)
\lineto(235.8742981,869.3929817)
\lineto(240.3664856,852.3226692)
\lineto(244.83914185,869.3929817)
\lineto(248.43289185,869.3929817)
\lineto(242.7102356,847.5179817)
\lineto(238.47195435,847.5179817)
\lineto(233.7649231,865.4476692)
\lineto(229.0383606,847.5179817)
\lineto(224.80007935,847.5179817)
\lineto(219.0774231,869.3929817)
\closepath
}
}
{
\newrgbcolor{curcolor}{0 0 0}
\pscustom[linestyle=none,fillstyle=solid,fillcolor=curcolor]
{
\newpath
\moveto(270.5227356,840.8773567)
\lineto(270.5227356,838.08438795)
\lineto(249.7414856,838.08438795)
\lineto(249.7414856,840.8773567)
\lineto(270.5227356,840.8773567)
\closepath
}
}
{
\newrgbcolor{curcolor}{0 0 0}
\pscustom[linestyle=none,fillstyle=solid,fillcolor=curcolor]
{
\newpath
\moveto(273.90164185,869.3929817)
\lineto(277.49539185,869.3929817)
\lineto(277.49539185,847.5179817)
\lineto(273.90164185,847.5179817)
\lineto(273.90164185,869.3929817)
\closepath
\moveto(273.90164185,877.9086067)
\lineto(277.49539185,877.9086067)
\lineto(277.49539185,873.35782545)
\lineto(273.90164185,873.35782545)
\lineto(273.90164185,877.9086067)
\closepath
}
}
{
\newrgbcolor{curcolor}{0 0 0}
\pscustom[linestyle=none,fillstyle=solid,fillcolor=curcolor]
{
\newpath
\moveto(298.19851685,865.6820442)
\lineto(323.23757935,865.6820442)
\lineto(323.23757935,862.4007942)
\lineto(298.19851685,862.4007942)
\lineto(298.19851685,865.6820442)
\closepath
\moveto(298.19851685,857.7132942)
\lineto(323.23757935,857.7132942)
\lineto(323.23757935,854.3929817)
\lineto(298.19851685,854.3929817)
\lineto(298.19851685,857.7132942)
\closepath
}
}
{
\newrgbcolor{curcolor}{0 0 0}
\pscustom[linestyle=none,fillstyle=solid,fillcolor=curcolor]
{
\newpath
\moveto(345.1711731,850.8382942)
\lineto(351.6164856,850.8382942)
\lineto(351.6164856,873.08438795)
\lineto(344.60476685,871.67813795)
\lineto(344.60476685,875.27188795)
\lineto(351.5774231,876.67813795)
\lineto(355.5227356,876.67813795)
\lineto(355.5227356,850.8382942)
\lineto(361.9680481,850.8382942)
\lineto(361.9680481,847.5179817)
\lineto(345.1711731,847.5179817)
\lineto(345.1711731,850.8382942)
\closepath
}
}
{
\newrgbcolor{curcolor}{0 0 0}
\pscustom[linestyle=none,fillstyle=solid,fillcolor=curcolor]
{
\newpath
\moveto(379.1946106,841.63907545)
\lineto(377.2414856,841.63907545)
\lineto(377.22195435,847.5179817)
\curveto(375.85476685,847.54402337)(374.48757935,847.70027337)(373.12039185,847.9867317)
\curveto(371.75320435,848.28621087)(370.37950643,848.7289192)(368.9992981,849.3148567)
\lineto(368.9992981,852.8304817)
\curveto(370.3274231,851.99714837)(371.66856893,851.36563795)(373.0227356,850.93595045)
\curveto(374.3899231,850.51928379)(375.7961731,850.30444004)(377.2414856,850.2914192)
\lineto(377.2414856,859.1976692)
\curveto(374.36388143,859.6664192)(372.26752726,860.46069004)(370.9524231,861.5804817)
\curveto(369.65033976,862.70027337)(368.9992981,864.2367317)(368.9992981,866.1898567)
\curveto(368.9992981,868.31225254)(369.70893351,869.98542962)(371.12820435,871.20938795)
\curveto(372.54747518,872.43334629)(374.5852356,873.13647129)(377.2414856,873.31876295)
\lineto(377.2414856,877.9086067)
\lineto(379.1946106,877.9086067)
\lineto(379.1946106,873.3773567)
\curveto(380.4055481,873.32527337)(381.5774231,873.19506504)(382.7102356,872.9867317)
\curveto(383.8430481,872.7914192)(384.94981893,872.5179817)(386.0305481,872.1664192)
\lineto(386.0305481,868.74845045)
\curveto(384.94981893,869.29532545)(383.83653768,869.71850254)(382.69070435,870.0179817)
\curveto(381.55789185,870.31746087)(380.39252726,870.49324212)(379.1946106,870.54532545)
\lineto(379.1946106,862.2054817)
\curveto(382.15033976,861.74975254)(384.32481893,860.93595045)(385.7180481,859.76407545)
\curveto(387.11127726,858.59220045)(387.80789185,856.99063795)(387.80789185,854.95938795)
\curveto(387.80789185,852.75886712)(387.06570435,851.02058587)(385.58132935,849.7445442)
\curveto(384.10997518,848.48152337)(381.98106893,847.7523567)(379.1946106,847.5570442)
\lineto(379.1946106,841.63907545)
\closepath
\moveto(377.2414856,862.5570442)
\lineto(377.2414856,870.5648567)
\curveto(375.73106893,870.39558587)(374.57872518,869.96589837)(373.78445435,869.2757942)
\curveto(372.99018351,868.58569004)(372.5930481,867.66772129)(372.5930481,866.52188795)
\curveto(372.5930481,865.40209629)(372.95763143,864.52970045)(373.6867981,863.90470045)
\curveto(374.4289856,863.27970045)(375.61388143,862.8304817)(377.2414856,862.5570442)
\closepath
\moveto(379.1946106,858.8070442)
\lineto(379.1946106,850.35001295)
\curveto(380.84825643,850.57136712)(382.09174601,851.04011712)(382.92507935,851.75626295)
\curveto(383.77143351,852.47240879)(384.1946106,853.4164192)(384.1946106,854.5882942)
\curveto(384.1946106,855.73412754)(383.79096476,856.64558587)(382.9836731,857.3226692)
\curveto(382.18940226,857.99975254)(380.92638143,858.4945442)(379.1946106,858.8070442)
\closepath
}
}
{
\newrgbcolor{curcolor}{0 0 0}
\pscustom[linestyle=none,fillstyle=solid,fillcolor=curcolor]
{
\newpath
\moveto(343.83551025,909.31929701)
\lineto(341.88238525,909.31929701)
\lineto(341.862854,915.19820326)
\curveto(340.4956665,915.22424493)(339.128479,915.38049493)(337.7612915,915.66695326)
\curveto(336.394104,915.96643243)(335.02040609,916.40914076)(333.64019775,916.99507826)
\lineto(333.64019775,920.51070326)
\curveto(334.96832275,919.67736993)(336.30946859,919.04585951)(337.66363525,918.61617201)
\curveto(339.03082275,918.19950534)(340.43707275,917.98466159)(341.88238525,917.97164076)
\lineto(341.88238525,926.87789076)
\curveto(339.00478109,927.34664076)(336.90842692,928.14091159)(335.59332275,929.26070326)
\curveto(334.29123942,930.38049493)(333.64019775,931.91695326)(333.64019775,933.87007826)
\curveto(333.64019775,935.99247409)(334.34983317,937.66565118)(335.769104,938.88960951)
\curveto(337.18837484,940.11356784)(339.22613525,940.81669284)(341.88238525,940.99898451)
\lineto(341.88238525,945.58882826)
\lineto(343.83551025,945.58882826)
\lineto(343.83551025,941.05757826)
\curveto(345.04644775,941.00549493)(346.21832275,940.87528659)(347.35113525,940.66695326)
\curveto(348.48394775,940.47164076)(349.59071859,940.19820326)(350.67144775,939.84664076)
\lineto(350.67144775,936.42867201)
\curveto(349.59071859,936.97554701)(348.47743734,937.39872409)(347.331604,937.69820326)
\curveto(346.1987915,937.99768243)(345.03342692,938.17346368)(343.83551025,938.22554701)
\lineto(343.83551025,929.88570326)
\curveto(346.79123942,929.42997409)(348.96571859,928.61617201)(350.35894775,927.44429701)
\curveto(351.75217692,926.27242201)(352.4487915,924.67085951)(352.4487915,922.63960951)
\curveto(352.4487915,920.43908868)(351.706604,918.70080743)(350.222229,917.42476576)
\curveto(348.75087484,916.16174493)(346.62196859,915.43257826)(343.83551025,915.23726576)
\lineto(343.83551025,909.31929701)
\closepath
\moveto(341.88238525,930.23726576)
\lineto(341.88238525,938.24507826)
\curveto(340.37196859,938.07580743)(339.21962484,937.64611993)(338.425354,936.95601576)
\curveto(337.63108317,936.26591159)(337.23394775,935.34794284)(337.23394775,934.20210951)
\curveto(337.23394775,933.08231784)(337.59853109,932.20992201)(338.32769775,931.58492201)
\curveto(339.06988525,930.95992201)(340.25478109,930.51070326)(341.88238525,930.23726576)
\closepath
\moveto(343.83551025,926.48726576)
\lineto(343.83551025,918.03023451)
\curveto(345.48915609,918.25158868)(346.73264567,918.72033868)(347.565979,919.43648451)
\curveto(348.41233317,920.15263034)(348.83551025,921.09664076)(348.83551025,922.26851576)
\curveto(348.83551025,923.41434909)(348.43186442,924.32580743)(347.62457275,925.00289076)
\curveto(346.83030192,925.67997409)(345.56728109,926.17476576)(343.83551025,926.48726576)
\closepath
}
}
{
\newrgbcolor{curcolor}{0 0 0}
\pscustom[linestyle=none,fillstyle=solid,fillcolor=curcolor]
{
\newpath
\moveto(359.10894775,944.35835951)
\lineto(369.26519775,911.48726576)
\lineto(365.94488525,911.48726576)
\lineto(355.78863525,944.35835951)
\lineto(359.10894775,944.35835951)
\closepath
}
}
{
\newrgbcolor{curcolor}{0 0 0}
\pscustom[linestyle=none,fillstyle=solid,fillcolor=curcolor]
{
\newpath
\moveto(370.4956665,937.07320326)
\lineto(374.30426025,937.07320326)
\lineto(381.14019775,918.71382826)
\lineto(387.97613525,937.07320326)
\lineto(391.784729,937.07320326)
\lineto(383.581604,915.19820326)
\lineto(378.6987915,915.19820326)
\lineto(370.4956665,937.07320326)
\closepath
}
}
{
\newrgbcolor{curcolor}{0 0 0}
\pscustom[linestyle=none,fillstyle=solid,fillcolor=curcolor]
{
\newpath
\moveto(406.68707275,926.19429701)
\curveto(403.78342692,926.19429701)(401.77170817,925.86226576)(400.6519165,925.19820326)
\curveto(399.53212484,924.53414076)(398.972229,923.40132826)(398.972229,921.79976576)
\curveto(398.972229,920.52372409)(399.38889567,919.50809909)(400.222229,918.75289076)
\curveto(401.06858317,918.01070326)(402.2144165,917.63960951)(403.659729,917.63960951)
\curveto(405.6519165,917.63960951)(407.24696859,918.34273451)(408.44488525,919.74898451)
\curveto(409.65582275,921.16825534)(410.2612915,923.04976576)(410.2612915,925.39351576)
\lineto(410.2612915,926.19429701)
\lineto(406.68707275,926.19429701)
\closepath
\moveto(413.8550415,927.67867201)
\lineto(413.8550415,915.19820326)
\lineto(410.2612915,915.19820326)
\lineto(410.2612915,918.51851576)
\curveto(409.440979,917.19039076)(408.41884359,916.20731784)(407.19488525,915.56929701)
\curveto(405.97092692,914.94429701)(404.47353109,914.63179701)(402.70269775,914.63179701)
\curveto(400.46311442,914.63179701)(398.67926025,915.25679701)(397.35113525,916.50679701)
\curveto(396.03603109,917.76981784)(395.378479,919.45601576)(395.378479,921.56539076)
\curveto(395.378479,924.02632826)(396.1987915,925.88179701)(397.8394165,927.13179701)
\curveto(399.49306234,928.38179701)(401.95399984,929.00679701)(405.222229,929.00679701)
\lineto(410.2612915,929.00679701)
\lineto(410.2612915,929.35835951)
\curveto(410.2612915,931.01200534)(409.7144165,932.28804701)(408.6206665,933.18648451)
\curveto(407.53993734,934.09794284)(406.01649984,934.55367201)(404.050354,934.55367201)
\curveto(402.800354,934.55367201)(401.58290609,934.40393243)(400.39801025,934.10445326)
\curveto(399.21311442,933.80497409)(398.0737915,933.35575534)(396.9800415,932.75679701)
\lineto(396.9800415,936.07710951)
\curveto(398.29514567,936.58492201)(399.57118734,936.96252618)(400.8081665,937.20992201)
\curveto(402.04514567,937.47033868)(403.24957275,937.60054701)(404.42144775,937.60054701)
\curveto(407.58551025,937.60054701)(409.9487915,936.78023451)(411.5112915,935.13960951)
\curveto(413.0737915,933.49898451)(413.8550415,931.01200534)(413.8550415,927.67867201)
\closepath
}
}
{
\newrgbcolor{curcolor}{0 0 0}
\pscustom[linestyle=none,fillstyle=solid,fillcolor=curcolor]
{
\newpath
\moveto(433.95269775,933.71382826)
\curveto(433.54905192,933.94820326)(433.10634359,934.11747409)(432.62457275,934.22164076)
\curveto(432.15582275,934.33882826)(431.63498942,934.39742201)(431.06207275,934.39742201)
\curveto(429.03082275,934.39742201)(427.46832275,933.73335951)(426.37457275,932.40523451)
\curveto(425.29384359,931.09013034)(424.753479,929.19559909)(424.753479,926.72164076)
\lineto(424.753479,915.19820326)
\lineto(421.14019775,915.19820326)
\lineto(421.14019775,937.07320326)
\lineto(424.753479,937.07320326)
\lineto(424.753479,933.67476576)
\curveto(425.50868734,935.00289076)(426.49176025,935.98596368)(427.70269775,936.62398451)
\curveto(428.91363525,937.27502618)(430.38498942,937.60054701)(432.11676025,937.60054701)
\curveto(432.36415609,937.60054701)(432.63759359,937.58101576)(432.93707275,937.54195326)
\curveto(433.23655192,937.51591159)(433.56858317,937.47033868)(433.9331665,937.40523451)
\lineto(433.95269775,933.71382826)
\closepath
}
}
{
\newrgbcolor{curcolor}{0 0 0}
\pscustom[linestyle=none,fillstyle=solid,fillcolor=curcolor]
{
\newpath
\moveto(441.237854,918.47945326)
\lineto(441.237854,906.87789076)
\lineto(437.62457275,906.87789076)
\lineto(437.62457275,937.07320326)
\lineto(441.237854,937.07320326)
\lineto(441.237854,933.75289076)
\curveto(441.99306234,935.05497409)(442.94358317,936.01851576)(444.0894165,936.64351576)
\curveto(445.24827067,937.28153659)(446.628479,937.60054701)(448.2300415,937.60054701)
\curveto(450.8862915,937.60054701)(453.04123942,936.54585951)(454.69488525,934.43648451)
\curveto(456.36155192,932.32710951)(457.19488525,929.55367201)(457.19488525,926.11617201)
\curveto(457.19488525,922.67867201)(456.36155192,919.90523451)(454.69488525,917.79585951)
\curveto(453.04123942,915.68648451)(450.8862915,914.63179701)(448.2300415,914.63179701)
\curveto(446.628479,914.63179701)(445.24827067,914.94429701)(444.0894165,915.56929701)
\curveto(442.94358317,916.20731784)(441.99306234,917.17736993)(441.237854,918.47945326)
\closepath
\moveto(453.4644165,926.11617201)
\curveto(453.4644165,928.75940118)(452.9175415,930.82971368)(451.8237915,932.32710951)
\curveto(450.74306234,933.83752618)(449.25217692,934.59273451)(447.35113525,934.59273451)
\curveto(445.45009359,934.59273451)(443.95269775,933.83752618)(442.85894775,932.32710951)
\curveto(441.77821859,930.82971368)(441.237854,928.75940118)(441.237854,926.11617201)
\curveto(441.237854,923.47294284)(441.77821859,921.39611993)(442.85894775,919.88570326)
\curveto(443.95269775,918.38830743)(445.45009359,917.63960951)(447.35113525,917.63960951)
\curveto(449.25217692,917.63960951)(450.74306234,918.38830743)(451.8237915,919.88570326)
\curveto(452.9175415,921.39611993)(453.4644165,923.47294284)(453.4644165,926.11617201)
\closepath
}
}
{
\newrgbcolor{curcolor}{0 0 0}
\pscustom[linestyle=none,fillstyle=solid,fillcolor=curcolor]
{
\newpath
\moveto(481.33551025,928.40132826)
\lineto(481.33551025,915.19820326)
\lineto(477.74176025,915.19820326)
\lineto(477.74176025,928.28414076)
\curveto(477.74176025,930.35445326)(477.33811442,931.90393243)(476.53082275,932.93257826)
\curveto(475.72353109,933.96122409)(474.51259359,934.47554701)(472.89801025,934.47554701)
\curveto(470.95790609,934.47554701)(469.42795817,933.85705743)(468.3081665,932.62007826)
\curveto(467.18837484,931.38309909)(466.628479,929.69690118)(466.628479,927.56148451)
\lineto(466.628479,915.19820326)
\lineto(463.01519775,915.19820326)
\lineto(463.01519775,945.58882826)
\lineto(466.628479,945.58882826)
\lineto(466.628479,933.67476576)
\curveto(467.487854,934.98986993)(468.49696859,935.97294284)(469.65582275,936.62398451)
\curveto(470.82769775,937.27502618)(472.175354,937.60054701)(473.6987915,937.60054701)
\curveto(476.21181234,937.60054701)(478.112854,936.81929701)(479.4019165,935.25679701)
\curveto(480.690979,933.70731784)(481.33551025,931.42216159)(481.33551025,928.40132826)
\closepath
}
}
{
\newrgbcolor{curcolor}{0 0 0}
\pscustom[linestyle=none,fillstyle=solid,fillcolor=curcolor]
{
\newpath
\moveto(488.5425415,937.07320326)
\lineto(492.1362915,937.07320326)
\lineto(492.1362915,915.19820326)
\lineto(488.5425415,915.19820326)
\lineto(488.5425415,937.07320326)
\closepath
\moveto(488.5425415,945.58882826)
\lineto(492.1362915,945.58882826)
\lineto(492.1362915,941.03804701)
\lineto(488.5425415,941.03804701)
\lineto(488.5425415,945.58882826)
\closepath
}
}
{
\newrgbcolor{curcolor}{0 0 0}
\pscustom[linestyle=none,fillstyle=solid,fillcolor=curcolor]
{
\newpath
\moveto(512.8394165,933.36226576)
\lineto(537.878479,933.36226576)
\lineto(537.878479,930.08101576)
\lineto(512.8394165,930.08101576)
\lineto(512.8394165,933.36226576)
\closepath
\moveto(512.8394165,925.39351576)
\lineto(537.878479,925.39351576)
\lineto(537.878479,922.07320326)
\lineto(512.8394165,922.07320326)
\lineto(512.8394165,925.39351576)
\closepath
}
}
{
\newrgbcolor{curcolor}{0 0 0}
\pscustom[linestyle=none,fillstyle=solid,fillcolor=curcolor]
{
\newpath
\moveto(572.565979,936.42867201)
\lineto(572.565979,933.03023451)
\curveto(571.550354,933.55106784)(570.4956665,933.94169284)(569.4019165,934.20210951)
\curveto(568.3081665,934.46252618)(567.175354,934.59273451)(566.003479,934.59273451)
\curveto(564.21962484,934.59273451)(562.878479,934.31929701)(561.9800415,933.77242201)
\curveto(561.09462484,933.22554701)(560.6519165,932.40523451)(560.6519165,931.31148451)
\curveto(560.6519165,930.47815118)(560.97092692,929.82059909)(561.60894775,929.33882826)
\curveto(562.24696859,928.87007826)(563.52952067,928.42085951)(565.456604,927.99117201)
\lineto(566.68707275,927.71773451)
\curveto(569.23915609,927.17085951)(571.04905192,926.39611993)(572.11676025,925.39351576)
\curveto(573.19748942,924.40393243)(573.737854,923.01721368)(573.737854,921.23335951)
\curveto(573.737854,919.20210951)(572.93056234,917.59403659)(571.315979,916.40914076)
\curveto(569.7144165,915.22424493)(567.50738525,914.63179701)(564.69488525,914.63179701)
\curveto(563.52301025,914.63179701)(562.29905192,914.74898451)(561.02301025,914.98335951)
\curveto(559.75998942,915.20471368)(558.425354,915.54325534)(557.019104,915.99898451)
\lineto(557.019104,919.70992201)
\curveto(558.347229,919.01981784)(559.65582275,918.49898451)(560.94488525,918.14742201)
\curveto(562.23394775,917.80888034)(563.50998942,917.63960951)(564.77301025,917.63960951)
\curveto(566.46571859,917.63960951)(567.76780192,917.92606784)(568.67926025,918.49898451)
\curveto(569.59071859,919.08492201)(570.04644775,919.90523451)(570.04644775,920.95992201)
\curveto(570.04644775,921.93648451)(569.7144165,922.68518243)(569.050354,923.20601576)
\curveto(568.39931234,923.72684909)(566.96051025,924.22815118)(564.73394775,924.70992201)
\lineto(563.48394775,925.00289076)
\curveto(561.25738525,925.47164076)(559.64931234,926.18778659)(558.659729,927.15132826)
\curveto(557.67014567,928.12789076)(557.175354,929.46252618)(557.175354,931.15523451)
\curveto(557.175354,933.21252618)(557.90452067,934.80106784)(559.362854,935.92085951)
\curveto(560.82118734,937.04065118)(562.89149984,937.60054701)(565.5737915,937.60054701)
\curveto(566.9019165,937.60054701)(568.1519165,937.50289076)(569.3237915,937.30757826)
\curveto(570.4956665,937.11226576)(571.57639567,936.81929701)(572.565979,936.42867201)
\closepath
}
}
{
\newrgbcolor{curcolor}{0 0 0}
\pscustom[linestyle=none,fillstyle=solid,fillcolor=curcolor]
{
\newpath
\moveto(593.87457275,926.38960951)
\curveto(593.87457275,928.99377618)(593.33420817,931.01200534)(592.253479,932.44429701)
\curveto(591.18577067,933.87658868)(589.68186442,934.59273451)(587.74176025,934.59273451)
\curveto(585.81467692,934.59273451)(584.31077067,933.87658868)(583.2300415,932.44429701)
\curveto(582.16233317,931.01200534)(581.628479,928.99377618)(581.628479,926.38960951)
\curveto(581.628479,923.79846368)(582.16233317,921.78674493)(583.2300415,920.35445326)
\curveto(584.31077067,918.92216159)(585.81467692,918.20601576)(587.74176025,918.20601576)
\curveto(589.68186442,918.20601576)(591.18577067,918.92216159)(592.253479,920.35445326)
\curveto(593.33420817,921.78674493)(593.87457275,923.79846368)(593.87457275,926.38960951)
\closepath
\moveto(597.46832275,917.91304701)
\curveto(597.46832275,914.18908868)(596.64149984,911.42216159)(594.987854,909.61226576)
\curveto(593.33420817,907.78934909)(590.80165609,906.87789076)(587.39019775,906.87789076)
\curveto(586.12717692,906.87789076)(584.93577067,906.97554701)(583.815979,907.17085951)
\curveto(582.69618734,907.35315118)(581.60894775,907.63960951)(580.55426025,908.03023451)
\lineto(580.55426025,911.52632826)
\curveto(581.60894775,910.95341159)(582.65061442,910.53023451)(583.67926025,910.25679701)
\curveto(584.70790609,909.98335951)(585.75608317,909.84664076)(586.8237915,909.84664076)
\curveto(589.18056234,909.84664076)(590.94488525,910.46513034)(592.11676025,911.70210951)
\curveto(593.28863525,912.92606784)(593.87457275,914.78153659)(593.87457275,917.26851576)
\lineto(593.87457275,919.04585951)
\curveto(593.13238525,917.75679701)(592.18186442,916.79325534)(591.02301025,916.15523451)
\curveto(589.86415609,915.51721368)(588.47743734,915.19820326)(586.862854,915.19820326)
\curveto(584.18056234,915.19820326)(582.019104,916.22033868)(580.378479,918.26460951)
\curveto(578.737854,920.30888034)(577.9175415,923.01721368)(577.9175415,926.38960951)
\curveto(577.9175415,929.77502618)(578.737854,932.48986993)(580.378479,934.53414076)
\curveto(582.019104,936.57841159)(584.18056234,937.60054701)(586.862854,937.60054701)
\curveto(588.47743734,937.60054701)(589.86415609,937.28153659)(591.02301025,936.64351576)
\curveto(592.18186442,936.00549493)(593.13238525,935.04195326)(593.87457275,933.75289076)
\lineto(593.87457275,937.07320326)
\lineto(597.46832275,937.07320326)
\lineto(597.46832275,917.91304701)
\closepath
}
}
{
\newrgbcolor{curcolor}{0 0 0}
\pscustom[linestyle=none,fillstyle=solid,fillcolor=curcolor]
{
\newpath
\moveto(623.05426025,928.40132826)
\lineto(623.05426025,915.19820326)
\lineto(619.46051025,915.19820326)
\lineto(619.46051025,928.28414076)
\curveto(619.46051025,930.35445326)(619.05686442,931.90393243)(618.24957275,932.93257826)
\curveto(617.44228109,933.96122409)(616.23134359,934.47554701)(614.61676025,934.47554701)
\curveto(612.67665609,934.47554701)(611.14670817,933.85705743)(610.0269165,932.62007826)
\curveto(608.90712484,931.38309909)(608.347229,929.69690118)(608.347229,927.56148451)
\lineto(608.347229,915.19820326)
\lineto(604.73394775,915.19820326)
\lineto(604.73394775,937.07320326)
\lineto(608.347229,937.07320326)
\lineto(608.347229,933.67476576)
\curveto(609.206604,934.98986993)(610.21571859,935.97294284)(611.37457275,936.62398451)
\curveto(612.54644775,937.27502618)(613.894104,937.60054701)(615.4175415,937.60054701)
\curveto(617.93056234,937.60054701)(619.831604,936.81929701)(621.1206665,935.25679701)
\curveto(622.409729,933.70731784)(623.05426025,931.42216159)(623.05426025,928.40132826)
\closepath
}
}
{
\newrgbcolor{curcolor}{0 0 0}
\pscustom[linestyle=none,fillstyle=solid,fillcolor=curcolor]
{
\newpath
\moveto(638.894104,945.54976576)
\curveto(637.14931234,942.55497409)(635.85373942,939.59273451)(635.00738525,936.66304701)
\curveto(634.16103109,933.73335951)(633.737854,930.76460951)(633.737854,927.75679701)
\curveto(633.737854,924.74898451)(634.16103109,921.76721368)(635.00738525,918.81148451)
\curveto(635.86676025,915.86877618)(637.16233317,912.90653659)(638.894104,909.92476576)
\lineto(635.769104,909.92476576)
\curveto(633.815979,912.98466159)(632.35113525,915.99247409)(631.37457275,918.94820326)
\curveto(630.41103109,921.90393243)(629.92926025,924.84013034)(629.92926025,927.75679701)
\curveto(629.92926025,930.66044284)(630.41103109,933.58361993)(631.37457275,936.52632826)
\curveto(632.33811442,939.46903659)(633.80295817,942.47684909)(635.769104,945.54976576)
\lineto(638.894104,945.54976576)
\closepath
}
}
{
\newrgbcolor{curcolor}{0 0 0}
\pscustom[linestyle=none,fillstyle=solid,fillcolor=curcolor]
{
\newpath
\moveto(645.31988525,945.54976576)
\lineto(648.44488525,945.54976576)
\curveto(650.39801025,942.47684909)(651.85634359,939.46903659)(652.81988525,936.52632826)
\curveto(653.79644775,933.58361993)(654.284729,930.66044284)(654.284729,927.75679701)
\curveto(654.284729,924.84013034)(653.79644775,921.90393243)(652.81988525,918.94820326)
\curveto(651.85634359,915.99247409)(650.39801025,912.98466159)(648.44488525,909.92476576)
\lineto(645.31988525,909.92476576)
\curveto(647.05165609,912.90653659)(648.34071859,915.86877618)(649.18707275,918.81148451)
\curveto(650.04644775,921.76721368)(650.47613525,924.74898451)(650.47613525,927.75679701)
\curveto(650.47613525,930.76460951)(650.04644775,933.73335951)(649.18707275,936.66304701)
\curveto(648.34071859,939.59273451)(647.05165609,942.55497409)(645.31988525,945.54976576)
\closepath
}
}
{
\newrgbcolor{curcolor}{0 0 0}
\pscustom[linestyle=none,fillstyle=solid,fillcolor=curcolor]
{
\newpath
\moveto(671.25738525,909.31929701)
\lineto(669.30426025,909.31929701)
\lineto(669.284729,915.19820326)
\curveto(667.9175415,915.22424493)(666.550354,915.38049493)(665.1831665,915.66695326)
\curveto(663.815979,915.96643243)(662.44228109,916.40914076)(661.06207275,916.99507826)
\lineto(661.06207275,920.51070326)
\curveto(662.39019775,919.67736993)(663.73134359,919.04585951)(665.08551025,918.61617201)
\curveto(666.45269775,918.19950534)(667.85894775,917.98466159)(669.30426025,917.97164076)
\lineto(669.30426025,926.87789076)
\curveto(666.42665609,927.34664076)(664.33030192,928.14091159)(663.01519775,929.26070326)
\curveto(661.71311442,930.38049493)(661.06207275,931.91695326)(661.06207275,933.87007826)
\curveto(661.06207275,935.99247409)(661.77170817,937.66565118)(663.190979,938.88960951)
\curveto(664.61024984,940.11356784)(666.64801025,940.81669284)(669.30426025,940.99898451)
\lineto(669.30426025,945.58882826)
\lineto(671.25738525,945.58882826)
\lineto(671.25738525,941.05757826)
\curveto(672.46832275,941.00549493)(673.64019775,940.87528659)(674.77301025,940.66695326)
\curveto(675.90582275,940.47164076)(677.01259359,940.19820326)(678.09332275,939.84664076)
\lineto(678.09332275,936.42867201)
\curveto(677.01259359,936.97554701)(675.89931234,937.39872409)(674.753479,937.69820326)
\curveto(673.6206665,937.99768243)(672.45530192,938.17346368)(671.25738525,938.22554701)
\lineto(671.25738525,929.88570326)
\curveto(674.21311442,929.42997409)(676.38759359,928.61617201)(677.78082275,927.44429701)
\curveto(679.17405192,926.27242201)(679.8706665,924.67085951)(679.8706665,922.63960951)
\curveto(679.8706665,920.43908868)(679.128479,918.70080743)(677.644104,917.42476576)
\curveto(676.17274984,916.16174493)(674.04384359,915.43257826)(671.25738525,915.23726576)
\lineto(671.25738525,909.31929701)
\closepath
\moveto(669.30426025,930.23726576)
\lineto(669.30426025,938.24507826)
\curveto(667.79384359,938.07580743)(666.64149984,937.64611993)(665.847229,936.95601576)
\curveto(665.05295817,936.26591159)(664.65582275,935.34794284)(664.65582275,934.20210951)
\curveto(664.65582275,933.08231784)(665.02040609,932.20992201)(665.74957275,931.58492201)
\curveto(666.49176025,930.95992201)(667.67665609,930.51070326)(669.30426025,930.23726576)
\closepath
\moveto(671.25738525,926.48726576)
\lineto(671.25738525,918.03023451)
\curveto(672.91103109,918.25158868)(674.15452067,918.72033868)(674.987854,919.43648451)
\curveto(675.83420817,920.15263034)(676.25738525,921.09664076)(676.25738525,922.26851576)
\curveto(676.25738525,923.41434909)(675.85373942,924.32580743)(675.04644775,925.00289076)
\curveto(674.25217692,925.67997409)(672.98915609,926.17476576)(671.25738525,926.48726576)
\closepath
}
}
{
\newrgbcolor{curcolor}{0 0 0}
\pscustom[linestyle=none,fillstyle=solid,fillcolor=curcolor]
{
\newpath
\moveto(238.77964783,908.30914991)
\lineto(236.82652283,908.30914991)
\lineto(236.80699158,914.18805616)
\curveto(235.43980408,914.21409783)(234.07261658,914.37034783)(232.70542908,914.65680616)
\curveto(231.33824158,914.95628533)(229.96454366,915.39899366)(228.58433533,915.98493116)
\lineto(228.58433533,919.50055616)
\curveto(229.91246033,918.66722283)(231.25360616,918.03571241)(232.60777283,917.60602491)
\curveto(233.97496033,917.18935825)(235.38121033,916.9745145)(236.82652283,916.96149366)
\lineto(236.82652283,925.86774366)
\curveto(233.94891866,926.33649366)(231.85256449,927.1307645)(230.53746033,928.25055616)
\curveto(229.23537699,929.37034783)(228.58433533,930.90680616)(228.58433533,932.85993116)
\curveto(228.58433533,934.982327)(229.29397074,936.65550408)(230.71324158,937.87946241)
\curveto(232.13251241,939.10342075)(234.17027283,939.80654575)(236.82652283,939.98883741)
\lineto(236.82652283,944.57868116)
\lineto(238.77964783,944.57868116)
\lineto(238.77964783,940.04743116)
\curveto(239.99058533,939.99534783)(241.16246033,939.8651395)(242.29527283,939.65680616)
\curveto(243.42808533,939.46149366)(244.53485616,939.18805616)(245.61558533,938.83649366)
\lineto(245.61558533,935.41852491)
\curveto(244.53485616,935.96539991)(243.42157491,936.388577)(242.27574158,936.68805616)
\curveto(241.14292908,936.98753533)(239.97756449,937.16331658)(238.77964783,937.21539991)
\lineto(238.77964783,928.87555616)
\curveto(241.73537699,928.419827)(243.90985616,927.60602491)(245.30308533,926.43414991)
\curveto(246.69631449,925.26227491)(247.39292908,923.66071241)(247.39292908,921.62946241)
\curveto(247.39292908,919.42894158)(246.65074158,917.69066033)(245.16636658,916.41461866)
\curveto(243.69501241,915.15159783)(241.56610616,914.42243116)(238.77964783,914.22711866)
\lineto(238.77964783,908.30914991)
\closepath
\moveto(236.82652283,929.22711866)
\lineto(236.82652283,937.23493116)
\curveto(235.31610616,937.06566033)(234.16376241,936.63597283)(233.36949158,935.94586866)
\curveto(232.57522074,935.2557645)(232.17808533,934.33779575)(232.17808533,933.19196241)
\curveto(232.17808533,932.07217075)(232.54266866,931.19977491)(233.27183533,930.57477491)
\curveto(234.01402283,929.94977491)(235.19891866,929.50055616)(236.82652283,929.22711866)
\closepath
\moveto(238.77964783,925.47711866)
\lineto(238.77964783,917.02008741)
\curveto(240.43329366,917.24144158)(241.67678324,917.71019158)(242.51011658,918.42633741)
\curveto(243.35647074,919.14248325)(243.77964783,920.08649366)(243.77964783,921.25836866)
\curveto(243.77964783,922.404202)(243.37600199,923.31566033)(242.56871033,923.99274366)
\curveto(241.77443949,924.669827)(240.51141866,925.16461866)(238.77964783,925.47711866)
\closepath
}
}
{
\newrgbcolor{curcolor}{0 0 0}
\pscustom[linestyle=none,fillstyle=solid,fillcolor=curcolor]
{
\newpath
\moveto(254.05308533,943.34821241)
\lineto(264.20933533,910.47711866)
\lineto(260.88902283,910.47711866)
\lineto(250.73277283,943.34821241)
\lineto(254.05308533,943.34821241)
\closepath
}
}
{
\newrgbcolor{curcolor}{0 0 0}
\pscustom[linestyle=none,fillstyle=solid,fillcolor=curcolor]
{
\newpath
\moveto(281.96324158,935.41852491)
\lineto(281.96324158,932.02008741)
\curveto(280.94761658,932.54092075)(279.89292908,932.93154575)(278.79917908,933.19196241)
\curveto(277.70542908,933.45237908)(276.57261658,933.58258741)(275.40074158,933.58258741)
\curveto(273.61688741,933.58258741)(272.27574158,933.30914991)(271.37730408,932.76227491)
\curveto(270.49188741,932.21539991)(270.04917908,931.39508741)(270.04917908,930.30133741)
\curveto(270.04917908,929.46800408)(270.36818949,928.810452)(271.00621033,928.32868116)
\curveto(271.64423116,927.85993116)(272.92678324,927.41071241)(274.85386658,926.98102491)
\lineto(276.08433533,926.70758741)
\curveto(278.63641866,926.16071241)(280.44631449,925.38597283)(281.51402283,924.38336866)
\curveto(282.59475199,923.39378533)(283.13511658,922.00706658)(283.13511658,920.22321241)
\curveto(283.13511658,918.19196241)(282.32782491,916.5838895)(280.71324158,915.39899366)
\curveto(279.11167908,914.21409783)(276.90464783,913.62164991)(274.09214783,913.62164991)
\curveto(272.92027283,913.62164991)(271.69631449,913.73883741)(270.42027283,913.97321241)
\curveto(269.15725199,914.19456658)(267.82261658,914.53310825)(266.41636658,914.98883741)
\lineto(266.41636658,918.69977491)
\curveto(267.74449158,918.00967075)(269.05308533,917.48883741)(270.34214783,917.13727491)
\curveto(271.63121033,916.79873325)(272.90725199,916.62946241)(274.17027283,916.62946241)
\curveto(275.86298116,916.62946241)(277.16506449,916.91592075)(278.07652283,917.48883741)
\curveto(278.98798116,918.07477491)(279.44371033,918.89508741)(279.44371033,919.94977491)
\curveto(279.44371033,920.92633741)(279.11167908,921.67503533)(278.44761658,922.19586866)
\curveto(277.79657491,922.716702)(276.35777283,923.21800408)(274.13121033,923.69977491)
\lineto(272.88121033,923.99274366)
\curveto(270.65464783,924.46149366)(269.04657491,925.1776395)(268.05699158,926.14118116)
\curveto(267.06740824,927.11774366)(266.57261658,928.45237908)(266.57261658,930.14508741)
\curveto(266.57261658,932.20237908)(267.30178324,933.79092075)(268.76011658,934.91071241)
\curveto(270.21844991,936.03050408)(272.28876241,936.59039991)(274.97105408,936.59039991)
\curveto(276.29917908,936.59039991)(277.54917908,936.49274366)(278.72105408,936.29743116)
\curveto(279.89292908,936.10211866)(280.97365824,935.80914991)(281.96324158,935.41852491)
\closepath
}
}
{
\newrgbcolor{curcolor}{0 0 0}
\pscustom[linestyle=none,fillstyle=solid,fillcolor=curcolor]
{
\newpath
\moveto(288.50621033,922.82086866)
\lineto(288.50621033,936.06305616)
\lineto(292.09996033,936.06305616)
\lineto(292.09996033,922.95758741)
\curveto(292.09996033,920.88727491)(292.50360616,919.33128533)(293.31089783,918.28961866)
\curveto(294.11818949,917.26097283)(295.32912699,916.74664991)(296.94371033,916.74664991)
\curveto(298.88381449,916.74664991)(300.41376241,917.3651395)(301.53355408,918.60211866)
\curveto(302.66636658,919.83909783)(303.23277283,921.52529575)(303.23277283,923.66071241)
\lineto(303.23277283,936.06305616)
\lineto(306.82652283,936.06305616)
\lineto(306.82652283,914.18805616)
\lineto(303.23277283,914.18805616)
\lineto(303.23277283,917.54743116)
\curveto(302.36037699,916.21930616)(301.34475199,915.22972283)(300.18589783,914.57868116)
\curveto(299.04006449,913.94066033)(297.70542908,913.62164991)(296.18199158,913.62164991)
\curveto(293.66897074,913.62164991)(291.76141866,914.40289991)(290.45933533,915.96539991)
\curveto(289.15725199,917.52789991)(288.50621033,919.81305616)(288.50621033,922.82086866)
\closepath
\moveto(297.54917908,936.59039991)
\lineto(297.54917908,936.59039991)
\closepath
}
}
{
\newrgbcolor{curcolor}{0 0 0}
\pscustom[linestyle=none,fillstyle=solid,fillcolor=curcolor]
{
\newpath
\moveto(331.29917908,931.86383741)
\curveto(332.19761658,933.47842075)(333.27183533,934.669827)(334.52183533,935.43805616)
\curveto(335.77183533,936.20628533)(337.24318949,936.59039991)(338.93589783,936.59039991)
\curveto(341.21454366,936.59039991)(342.97235616,935.78961866)(344.20933533,934.18805616)
\curveto(345.44631449,932.5995145)(346.06480408,930.3338895)(346.06480408,927.39118116)
\lineto(346.06480408,914.18805616)
\lineto(342.45152283,914.18805616)
\lineto(342.45152283,927.27399366)
\curveto(342.45152283,929.37034783)(342.08042908,930.92633741)(341.33824158,931.94196241)
\curveto(340.59605408,932.95758741)(339.46324158,933.46539991)(337.93980408,933.46539991)
\curveto(336.07782491,933.46539991)(334.60647074,932.84691033)(333.52574158,931.60993116)
\curveto(332.44501241,930.372952)(331.90464783,928.68675408)(331.90464783,926.55133741)
\lineto(331.90464783,914.18805616)
\lineto(328.29136658,914.18805616)
\lineto(328.29136658,927.27399366)
\curveto(328.29136658,929.38336866)(327.92027283,930.93935825)(327.17808533,931.94196241)
\curveto(326.43589783,932.95758741)(325.29006449,933.46539991)(323.74058533,933.46539991)
\curveto(321.90464783,933.46539991)(320.44631449,932.84039991)(319.36558533,931.59039991)
\curveto(318.28485616,930.35342075)(317.74449158,928.67373325)(317.74449158,926.55133741)
\lineto(317.74449158,914.18805616)
\lineto(314.13121033,914.18805616)
\lineto(314.13121033,936.06305616)
\lineto(317.74449158,936.06305616)
\lineto(317.74449158,932.66461866)
\curveto(318.56480408,934.0057645)(319.54787699,934.99534783)(320.69371033,935.63336866)
\curveto(321.83954366,936.2713895)(323.20022074,936.59039991)(324.77574158,936.59039991)
\curveto(326.36428324,936.59039991)(327.71193949,936.18675408)(328.81871033,935.37946241)
\curveto(329.93850199,934.57217075)(330.76532491,933.40029575)(331.29917908,931.86383741)
\closepath
}
}
\end{pspicture}
 
\end{figure}

\textbf{2. Arbitrary vector:} The weights depend the elements of $c$. $w_i$ is $\frac{1}{\sum_{i}c_i}$ for all $i$ where $c_i$ is 1 and -1 for all other $w_i$. The activation is again a binary step resulting 1 for $\tilde{z} = 1$ and 0 otherwise.

\begin{figure}[htb]
	\def\svgwidth{\textwidth}
	\input{drawing-2.pdf_tex} 
\end{figure}

\newpage

\begin{figure}[htb!]
\def\svgwidth{\textwidth}
\input{decision_boundaries.pdf_tex} 
\end{figure}
Generalization to higher dimensions: Draw hyperdimensional decision boundaries each dividing the input space into two subspaces. The number of boundaries relates to the dimension of the hypercube.

\section*{2.2 3-layer Universal Classifier}

The first layer are the discriminative hyperplanes from 2.1.3 that divide the input space into two subgroups. Layer two employs the Arbitrary vector neurons that each decide for one corner of the hypercube whether the data point belongs to the group. The third layer then combines the results from the corners with the same class using the OR neurons from 2.1.1.

\begin{figure}[htb!]
	\def\svgwidth{\textwidth}
	\input{network.pdf_tex} 
\end{figure}

Problems we see with these zero loss training classifiers are that they may be over-fitted to the training data.

\newpage

\section*{3.1 Linear Activation Function}
Using any linear activation function $\varphi(x) = x$ leads to the following outputs:
\begin{align*}
z_1 = \varphi(B_1 z_0) = B_1z_0\\
z_2 = B_2z_1 = B_2B_1z_0
\end{align*}
This could be replaced by a single layer with the parameters $B = B_1B_2$.

\section*{3.2 Weight Decay}
1. \begin{align*}
Loss(w) &= Loss_0(w)+\frac{\lambda}{2N}w^{\tau}w\\
\frac{\partial Loss}{\partial w} &= \frac{\partial}{\partial w} Loss_0 + \frac{\lambda}{N}w\\
w &= w-\tau \frac{\partial}{\partial w}Loss_0 - \tau \frac{\lambda}{N}w \\
&= (1-\frac{\tau\lambda}{N})w -\tau \frac{\partial}{\partial w}Loss_0\\
\rightarrow \epsilon &= \frac{\tau\lambda}{N}
\end{align*}
2. The weight decays in proportion to its size. Thus, larger weights are penalized and weights with a small magnitude are preferred which avoids overfitting. \\

\noindent 3. \begin{align*}
Loss(w) &= Loss_0(w)+\frac{\lambda}{2N}|w|\\
\frac{\partial Loss}{\partial w}& = \frac{\partial}{\partial w} Loss_0 + \frac{\lambda}{N}sgn(w)\\
w &= w-\tau \frac{\partial}{\partial w}Loss_0 - \tau \frac{\lambda}{N}sgn(w) 
\end{align*}

\noindent 4. Since the biases are fixed and representing the offset, not the curvature of the model, the regularization has little effect on them.

\section*{4. Application}

Honestly, very complicated object structure.

\end{document}